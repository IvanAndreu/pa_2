% Options for packages loaded elsewhere
\PassOptionsToPackage{unicode}{hyperref}
\PassOptionsToPackage{hyphens}{url}
%
\documentclass[
]{article}
\title{Readme}
\author{}
\date{\vspace{-2.5em}}

\usepackage{amsmath,amssymb}
\usepackage{lmodern}
\usepackage{iftex}
\ifPDFTeX
  \usepackage[T1]{fontenc}
  \usepackage[utf8]{inputenc}
  \usepackage{textcomp} % provide euro and other symbols
\else % if luatex or xetex
  \usepackage{unicode-math}
  \defaultfontfeatures{Scale=MatchLowercase}
  \defaultfontfeatures[\rmfamily]{Ligatures=TeX,Scale=1}
\fi
% Use upquote if available, for straight quotes in verbatim environments
\IfFileExists{upquote.sty}{\usepackage{upquote}}{}
\IfFileExists{microtype.sty}{% use microtype if available
  \usepackage[]{microtype}
  \UseMicrotypeSet[protrusion]{basicmath} % disable protrusion for tt fonts
}{}
\makeatletter
\@ifundefined{KOMAClassName}{% if non-KOMA class
  \IfFileExists{parskip.sty}{%
    \usepackage{parskip}
  }{% else
    \setlength{\parindent}{0pt}
    \setlength{\parskip}{6pt plus 2pt minus 1pt}}
}{% if KOMA class
  \KOMAoptions{parskip=half}}
\makeatother
\usepackage{xcolor}
\IfFileExists{xurl.sty}{\usepackage{xurl}}{} % add URL line breaks if available
\IfFileExists{bookmark.sty}{\usepackage{bookmark}}{\usepackage{hyperref}}
\hypersetup{
  pdftitle={Readme},
  hidelinks,
  pdfcreator={LaTeX via pandoc}}
\urlstyle{same} % disable monospaced font for URLs
\usepackage[margin=1in]{geometry}
\usepackage{color}
\usepackage{fancyvrb}
\newcommand{\VerbBar}{|}
\newcommand{\VERB}{\Verb[commandchars=\\\{\}]}
\DefineVerbatimEnvironment{Highlighting}{Verbatim}{commandchars=\\\{\}}
% Add ',fontsize=\small' for more characters per line
\usepackage{framed}
\definecolor{shadecolor}{RGB}{248,248,248}
\newenvironment{Shaded}{\begin{snugshade}}{\end{snugshade}}
\newcommand{\AlertTok}[1]{\textcolor[rgb]{0.94,0.16,0.16}{#1}}
\newcommand{\AnnotationTok}[1]{\textcolor[rgb]{0.56,0.35,0.01}{\textbf{\textit{#1}}}}
\newcommand{\AttributeTok}[1]{\textcolor[rgb]{0.77,0.63,0.00}{#1}}
\newcommand{\BaseNTok}[1]{\textcolor[rgb]{0.00,0.00,0.81}{#1}}
\newcommand{\BuiltInTok}[1]{#1}
\newcommand{\CharTok}[1]{\textcolor[rgb]{0.31,0.60,0.02}{#1}}
\newcommand{\CommentTok}[1]{\textcolor[rgb]{0.56,0.35,0.01}{\textit{#1}}}
\newcommand{\CommentVarTok}[1]{\textcolor[rgb]{0.56,0.35,0.01}{\textbf{\textit{#1}}}}
\newcommand{\ConstantTok}[1]{\textcolor[rgb]{0.00,0.00,0.00}{#1}}
\newcommand{\ControlFlowTok}[1]{\textcolor[rgb]{0.13,0.29,0.53}{\textbf{#1}}}
\newcommand{\DataTypeTok}[1]{\textcolor[rgb]{0.13,0.29,0.53}{#1}}
\newcommand{\DecValTok}[1]{\textcolor[rgb]{0.00,0.00,0.81}{#1}}
\newcommand{\DocumentationTok}[1]{\textcolor[rgb]{0.56,0.35,0.01}{\textbf{\textit{#1}}}}
\newcommand{\ErrorTok}[1]{\textcolor[rgb]{0.64,0.00,0.00}{\textbf{#1}}}
\newcommand{\ExtensionTok}[1]{#1}
\newcommand{\FloatTok}[1]{\textcolor[rgb]{0.00,0.00,0.81}{#1}}
\newcommand{\FunctionTok}[1]{\textcolor[rgb]{0.00,0.00,0.00}{#1}}
\newcommand{\ImportTok}[1]{#1}
\newcommand{\InformationTok}[1]{\textcolor[rgb]{0.56,0.35,0.01}{\textbf{\textit{#1}}}}
\newcommand{\KeywordTok}[1]{\textcolor[rgb]{0.13,0.29,0.53}{\textbf{#1}}}
\newcommand{\NormalTok}[1]{#1}
\newcommand{\OperatorTok}[1]{\textcolor[rgb]{0.81,0.36,0.00}{\textbf{#1}}}
\newcommand{\OtherTok}[1]{\textcolor[rgb]{0.56,0.35,0.01}{#1}}
\newcommand{\PreprocessorTok}[1]{\textcolor[rgb]{0.56,0.35,0.01}{\textit{#1}}}
\newcommand{\RegionMarkerTok}[1]{#1}
\newcommand{\SpecialCharTok}[1]{\textcolor[rgb]{0.00,0.00,0.00}{#1}}
\newcommand{\SpecialStringTok}[1]{\textcolor[rgb]{0.31,0.60,0.02}{#1}}
\newcommand{\StringTok}[1]{\textcolor[rgb]{0.31,0.60,0.02}{#1}}
\newcommand{\VariableTok}[1]{\textcolor[rgb]{0.00,0.00,0.00}{#1}}
\newcommand{\VerbatimStringTok}[1]{\textcolor[rgb]{0.31,0.60,0.02}{#1}}
\newcommand{\WarningTok}[1]{\textcolor[rgb]{0.56,0.35,0.01}{\textbf{\textit{#1}}}}
\usepackage{graphicx}
\makeatletter
\def\maxwidth{\ifdim\Gin@nat@width>\linewidth\linewidth\else\Gin@nat@width\fi}
\def\maxheight{\ifdim\Gin@nat@height>\textheight\textheight\else\Gin@nat@height\fi}
\makeatother
% Scale images if necessary, so that they will not overflow the page
% margins by default, and it is still possible to overwrite the defaults
% using explicit options in \includegraphics[width, height, ...]{}
\setkeys{Gin}{width=\maxwidth,height=\maxheight,keepaspectratio}
% Set default figure placement to htbp
\makeatletter
\def\fps@figure{htbp}
\makeatother
\setlength{\emergencystretch}{3em} % prevent overfull lines
\providecommand{\tightlist}{%
  \setlength{\itemsep}{0pt}\setlength{\parskip}{0pt}}
\setcounter{secnumdepth}{-\maxdimen} % remove section numbering
\ifLuaTeX
  \usepackage{selnolig}  % disable illegal ligatures
\fi

\begin{document}
\maketitle

\hypertarget{manipulate-the-dataframe-as-necessary-so-that-you-can-calculate-average-duration-f0-and-intensity-as-a-function-of-lexical-stress-extra-points-if-you-can-create-a-plot}{%
\section{Manipulate the dataframe as necessary so that you can calculate
average duration, f0 and intensity as a function of lexical stress
(extra points if you can create a
plot)}\label{manipulate-the-dataframe-as-necessary-so-that-you-can-calculate-average-duration-f0-and-intensity-as-a-function-of-lexical-stress-extra-points-if-you-can-create-a-plot}}

\hypertarget{load-library}{%
\subsection{Load library}\label{load-library}}

\begin{Shaded}
\begin{Highlighting}[]
\FunctionTok{library}\NormalTok{(tidyverse)}
\end{Highlighting}
\end{Shaded}

\begin{verbatim}
## -- Attaching packages --------------------------------------- tidyverse 1.3.1 --
\end{verbatim}

\begin{verbatim}
## v ggplot2 3.3.5     v purrr   0.3.4
## v tibble  3.1.6     v dplyr   1.0.7
## v tidyr   1.1.4     v stringr 1.4.0
## v readr   2.1.1     v forcats 0.5.1
\end{verbatim}

\begin{verbatim}
## -- Conflicts ------------------------------------------ tidyverse_conflicts() --
## x dplyr::filter() masks stats::filter()
## x dplyr::lag()    masks stats::lag()
\end{verbatim}

\hypertarget{read-data-into-r}{%
\subsection{Read data into R}\label{read-data-into-r}}

\begin{Shaded}
\begin{Highlighting}[]
\FunctionTok{read\_csv}\NormalTok{(}\StringTok{"data/data.csv"}\NormalTok{)}
\end{Highlighting}
\end{Shaded}

\begin{verbatim}
## Rows: 10 Columns: 4
\end{verbatim}

\begin{verbatim}
## -- Column specification --------------------------------------------------------
## Delimiter: ","
## chr (1): info
## dbl (3): durationV, f0, int
\end{verbatim}

\begin{verbatim}
## 
## i Use `spec()` to retrieve the full column specification for this data.
## i Specify the column types or set `show_col_types = FALSE` to quiet this message.
\end{verbatim}

\begin{verbatim}
## # A tibble: 10 x 4
##    info    durationV    f0   int
##    <chr>       <dbl> <dbl> <dbl>
##  1 capo_1       0.22  77.1  84.8
##  2 capo_2       0.07 127.   75.6
##  3 pinto_1      0.14 102.   86.8
##  4 pinto_2      0.09  84.8  78.9
##  5 pujo_1       0.08  90.7  82.6
##  6 pujo_2       0.06  79.2  73.7
##  7 quemo_1      0.16  78.6  77.6
##  8 quemo_2      0.09  82.5  74.6
##  9 testo_1      0.13  81.3  80.9
## 10 testo_2      0.07  77.8  76.9
\end{verbatim}

\hypertarget{creation-of-the-object-with-the-csv-file}{%
\subsection{Creation of the object with the CSV
file}\label{creation-of-the-object-with-the-csv-file}}

\begin{Shaded}
\begin{Highlighting}[]
\NormalTok{my\_data }\OtherTok{\textless{}{-}} \FunctionTok{read\_csv}\NormalTok{(}\StringTok{"data/data.csv"}\NormalTok{)}
\end{Highlighting}
\end{Shaded}

\begin{verbatim}
## Rows: 10 Columns: 4
\end{verbatim}

\begin{verbatim}
## -- Column specification --------------------------------------------------------
## Delimiter: ","
## chr (1): info
## dbl (3): durationV, f0, int
\end{verbatim}

\begin{verbatim}
## 
## i Use `spec()` to retrieve the full column specification for this data.
## i Specify the column types or set `show_col_types = FALSE` to quiet this message.
\end{verbatim}

\hypertarget{mean-duration}{%
\subsection{Mean duration}\label{mean-duration}}

\begin{Shaded}
\begin{Highlighting}[]
\FunctionTok{mean}\NormalTok{(my\_data}\SpecialCharTok{$}\NormalTok{durationV)}
\end{Highlighting}
\end{Shaded}

\begin{verbatim}
## [1] 0.111
\end{verbatim}

\hypertarget{mean-f0}{%
\subsection{Mean f0}\label{mean-f0}}

\begin{Shaded}
\begin{Highlighting}[]
\FunctionTok{mean}\NormalTok{(my\_data}\SpecialCharTok{$}\NormalTok{f0)}
\end{Highlighting}
\end{Shaded}

\begin{verbatim}
## [1] 88.076
\end{verbatim}

\hypertarget{mean-intensity}{%
\subsection{Mean intensity}\label{mean-intensity}}

\begin{Shaded}
\begin{Highlighting}[]
\FunctionTok{mean}\NormalTok{(my\_data}\SpecialCharTok{$}\NormalTok{int)}
\end{Highlighting}
\end{Shaded}

\begin{verbatim}
## [1] 79.254
\end{verbatim}

\hypertarget{summary}{%
\subsection{Summary}\label{summary}}

\begin{Shaded}
\begin{Highlighting}[]
\NormalTok{my\_data }\SpecialCharTok{\%\textgreater{}\%} 
  \FunctionTok{summarize}\NormalTok{(}\AttributeTok{dur\_avg =} \FunctionTok{mean}\NormalTok{(durationV), }\AttributeTok{f0\_avg =} \FunctionTok{mean}\NormalTok{(f0), }\AttributeTok{int\_avg =} \FunctionTok{mean}\NormalTok{(int))}
\end{Highlighting}
\end{Shaded}

\begin{verbatim}
## # A tibble: 1 x 3
##   dur_avg f0_avg int_avg
##     <dbl>  <dbl>   <dbl>
## 1   0.111   88.1    79.3
\end{verbatim}

Por alguna razón me redondea los números aquí

\hypertarget{separate-variable-info-into-different-columns-word-stress}{%
\subsection{Separate variable Info into different columns (word,
stress)}\label{separate-variable-info-into-different-columns-word-stress}}

\begin{Shaded}
\begin{Highlighting}[]
\FunctionTok{separate}\NormalTok{(}\AttributeTok{data =}\NormalTok{ my\_data, }\AttributeTok{col =}\NormalTok{ info, }\AttributeTok{into =} \FunctionTok{c}\NormalTok{(}\StringTok{"word"}\NormalTok{, }\StringTok{"stress"}\NormalTok{), }\AttributeTok{sep =} \StringTok{"\_"}\NormalTok{)}
\end{Highlighting}
\end{Shaded}

\begin{verbatim}
## # A tibble: 10 x 5
##    word  stress durationV    f0   int
##    <chr> <chr>      <dbl> <dbl> <dbl>
##  1 capo  1           0.22  77.1  84.8
##  2 capo  2           0.07 127.   75.6
##  3 pinto 1           0.14 102.   86.8
##  4 pinto 2           0.09  84.8  78.9
##  5 pujo  1           0.08  90.7  82.6
##  6 pujo  2           0.06  79.2  73.7
##  7 quemo 1           0.16  78.6  77.6
##  8 quemo 2           0.09  82.5  74.6
##  9 testo 1           0.13  81.3  80.9
## 10 testo 2           0.07  77.8  76.9
\end{verbatim}

\hypertarget{assign-it-to-an-object}{%
\subsection{Assign it to an object}\label{assign-it-to-an-object}}

\begin{Shaded}
\begin{Highlighting}[]
\NormalTok{my\_new\_data }\OtherTok{\textless{}{-}} \FunctionTok{separate}\NormalTok{(}\AttributeTok{data =}\NormalTok{ my\_data, }\AttributeTok{col =}\NormalTok{ info, }\AttributeTok{into =} \FunctionTok{c}\NormalTok{(}\StringTok{"word"}\NormalTok{, }\StringTok{"stress"}\NormalTok{), }\AttributeTok{sep =} \StringTok{"\_"}\NormalTok{)}
\end{Highlighting}
\end{Shaded}

\hypertarget{means-by-lexical-stress}{%
\subsection{Means by lexical stress}\label{means-by-lexical-stress}}

\hypertarget{means-of-durationv-for-stress-1-and-2}{%
\subsubsection{Means of durationV for stress 1 and
2}\label{means-of-durationv-for-stress-1-and-2}}

\begin{Shaded}
\begin{Highlighting}[]
\NormalTok{my\_new\_data }\SpecialCharTok{\%\textgreater{}\%}
  \FunctionTok{group\_by}\NormalTok{(stress) }\SpecialCharTok{\%\textgreater{}\%}
  \FunctionTok{summarize}\NormalTok{(}\AttributeTok{mean\_durationV =} \FunctionTok{mean}\NormalTok{(durationV))}
\end{Highlighting}
\end{Shaded}

\begin{verbatim}
## # A tibble: 2 x 2
##   stress mean_durationV
##   <chr>           <dbl>
## 1 1               0.146
## 2 2               0.076
\end{verbatim}

\hypertarget{plots}{%
\subsection{Plots}\label{plots}}

No se si esto esta bien

\hypertarget{plot-of-f0-and-stress}{%
\subsubsection{Plot of f0 and stress}\label{plot-of-f0-and-stress}}

\begin{Shaded}
\begin{Highlighting}[]
\NormalTok{my\_data }\SpecialCharTok{\%\textgreater{}\%} 
  \FunctionTok{separate}\NormalTok{(., }\AttributeTok{col =}\NormalTok{ info, }\AttributeTok{into =} \FunctionTok{c}\NormalTok{(}\StringTok{"item"}\NormalTok{, }\StringTok{"stress"}\NormalTok{), }\AttributeTok{sep =} \StringTok{"\_"}\NormalTok{) }\SpecialCharTok{\%\textgreater{}\%}
  \FunctionTok{ggplot}\NormalTok{(., }\FunctionTok{aes}\NormalTok{(}\AttributeTok{x =}\NormalTok{ stress, }\AttributeTok{y =}\NormalTok{ f0, }\AttributeTok{color =}\NormalTok{ item)) }\SpecialCharTok{+}
    \FunctionTok{geom\_point}\NormalTok{() }
\end{Highlighting}
\end{Shaded}

\includegraphics{README_files/figure-latex/unnamed-chunk-11-1.pdf}

\hypertarget{plot-of-intensity-and-stress}{%
\subsubsection{Plot of intensity and
stress}\label{plot-of-intensity-and-stress}}

\begin{Shaded}
\begin{Highlighting}[]
\NormalTok{my\_data }\SpecialCharTok{\%\textgreater{}\%} 
  \FunctionTok{separate}\NormalTok{(., }\AttributeTok{col =}\NormalTok{ info, }\AttributeTok{into =} \FunctionTok{c}\NormalTok{(}\StringTok{"item"}\NormalTok{, }\StringTok{"stress"}\NormalTok{), }\AttributeTok{sep =} \StringTok{"\_"}\NormalTok{) }\SpecialCharTok{\%\textgreater{}\%}
  \FunctionTok{ggplot}\NormalTok{(., }\FunctionTok{aes}\NormalTok{(}\AttributeTok{x =}\NormalTok{ stress, }\AttributeTok{y =}\NormalTok{ int, }\AttributeTok{color =}\NormalTok{ item)) }\SpecialCharTok{+}
    \FunctionTok{geom\_point}\NormalTok{() }
\end{Highlighting}
\end{Shaded}

\includegraphics{README_files/figure-latex/unnamed-chunk-12-1.pdf}

\hypertarget{plot-of-intensity-and-stress-1}{%
\subsubsection{Plot of intensity and
stress}\label{plot-of-intensity-and-stress-1}}

\begin{Shaded}
\begin{Highlighting}[]
\NormalTok{my\_new\_data }\SpecialCharTok{\%\textgreater{}\%}
  \FunctionTok{ggplot}\NormalTok{(., }\FunctionTok{aes}\NormalTok{(}\AttributeTok{x =}\NormalTok{ stress, }\AttributeTok{y =}\NormalTok{ int, }\AttributeTok{col =} \StringTok{"orange"}\NormalTok{))}\SpecialCharTok{+}
  \FunctionTok{geom\_point}\NormalTok{()}
\end{Highlighting}
\end{Shaded}

\includegraphics{README_files/figure-latex/unnamed-chunk-13-1.pdf}

\hypertarget{not-sure-what-am-i-doing}{%
\subsubsection{Not sure what am I
doing}\label{not-sure-what-am-i-doing}}

\begin{Shaded}
\begin{Highlighting}[]
\FunctionTok{plot}\NormalTok{(}\AttributeTok{x =}\NormalTok{ my\_data}\SpecialCharTok{$}\NormalTok{f0 , }\AttributeTok{y =}\NormalTok{ my\_data}\SpecialCharTok{$}\NormalTok{int, }\AttributeTok{type =} \StringTok{\textquotesingle{}o\textquotesingle{}}\NormalTok{, }\AttributeTok{xlab =} \StringTok{"f0"}\NormalTok{, }\AttributeTok{ylab =} \StringTok{"intensity"}\NormalTok{, }\AttributeTok{col =} \StringTok{"orange"}\NormalTok{)}
\end{Highlighting}
\end{Shaded}

\includegraphics{README_files/figure-latex/unnamed-chunk-14-1.pdf}

\begin{Shaded}
\begin{Highlighting}[]
\FunctionTok{plot}\NormalTok{(}\AttributeTok{x =}\NormalTok{ my\_new\_data}\SpecialCharTok{$}\NormalTok{f0 , }\AttributeTok{y =}\NormalTok{ my\_new\_data}\SpecialCharTok{$}\NormalTok{int, }\AttributeTok{type =} \StringTok{\textquotesingle{}o\textquotesingle{}}\NormalTok{, }\AttributeTok{xlab =} \StringTok{"f0"}\NormalTok{, }\AttributeTok{ylab =} \StringTok{"intensity"}\NormalTok{, }\AttributeTok{col =} \StringTok{"green"}\NormalTok{)}
\end{Highlighting}
\end{Shaded}

\includegraphics{README_files/figure-latex/unnamed-chunk-15-1.pdf}

\begin{Shaded}
\begin{Highlighting}[]
\FunctionTok{plot}\NormalTok{(}\AttributeTok{x =}\NormalTok{ my\_new\_data}\SpecialCharTok{$}\NormalTok{durationV , }\AttributeTok{y =}\NormalTok{ my\_new\_data}\SpecialCharTok{$}\NormalTok{int, }\AttributeTok{type =} \StringTok{\textquotesingle{}o\textquotesingle{}}\NormalTok{, }\AttributeTok{xlab =} \StringTok{"f0"}\NormalTok{, }\AttributeTok{ylab =} \StringTok{"intensity"}\NormalTok{, }\AttributeTok{col =} \StringTok{"green"}\NormalTok{)}
\end{Highlighting}
\end{Shaded}

\includegraphics{README_files/figure-latex/unnamed-chunk-16-1.pdf}

\begin{Shaded}
\begin{Highlighting}[]
\FunctionTok{hist}\NormalTok{(my\_data}\SpecialCharTok{$}\NormalTok{f0, }
     \AttributeTok{main=}\StringTok{"f0"}\NormalTok{, }
     \AttributeTok{xlab=}\StringTok{"f0"}\NormalTok{, }
     \AttributeTok{ylab =} \StringTok{"count"}\NormalTok{,}
     \AttributeTok{border=}\StringTok{"black"}\NormalTok{, }
     \AttributeTok{col=}\StringTok{"blue"}\NormalTok{,}
     \AttributeTok{xlim=}\FunctionTok{c}\NormalTok{(}\DecValTok{90}\NormalTok{,}\DecValTok{115}\NormalTok{),}
     \AttributeTok{las=}\DecValTok{2}\NormalTok{, }
     \AttributeTok{breaks=}\DecValTok{10}\NormalTok{)}
\end{Highlighting}
\end{Shaded}

\includegraphics{README_files/figure-latex/unnamed-chunk-17-1.pdf}

\end{document}
